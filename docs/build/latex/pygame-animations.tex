%% Generated by Sphinx.
\def\sphinxdocclass{report}
\documentclass[letterpaper,10pt,french]{sphinxmanual}
\ifdefined\pdfpxdimen
   \let\sphinxpxdimen\pdfpxdimen\else\newdimen\sphinxpxdimen
\fi \sphinxpxdimen=.75bp\relax

\PassOptionsToPackage{warn}{textcomp}
\usepackage[utf8]{inputenc}
\ifdefined\DeclareUnicodeCharacter
% support both utf8 and utf8x syntaxes
  \ifdefined\DeclareUnicodeCharacterAsOptional
    \def\sphinxDUC#1{\DeclareUnicodeCharacter{"#1}}
  \else
    \let\sphinxDUC\DeclareUnicodeCharacter
  \fi
  \sphinxDUC{00A0}{\nobreakspace}
  \sphinxDUC{2500}{\sphinxunichar{2500}}
  \sphinxDUC{2502}{\sphinxunichar{2502}}
  \sphinxDUC{2514}{\sphinxunichar{2514}}
  \sphinxDUC{251C}{\sphinxunichar{251C}}
  \sphinxDUC{2572}{\textbackslash}
\fi
\usepackage{cmap}
\usepackage[T1]{fontenc}
\usepackage{amsmath,amssymb,amstext}
\usepackage{babel}



\usepackage{times}
\expandafter\ifx\csname T@LGR\endcsname\relax
\else
% LGR was declared as font encoding
  \substitutefont{LGR}{\rmdefault}{cmr}
  \substitutefont{LGR}{\sfdefault}{cmss}
  \substitutefont{LGR}{\ttdefault}{cmtt}
\fi
\expandafter\ifx\csname T@X2\endcsname\relax
  \expandafter\ifx\csname T@T2A\endcsname\relax
  \else
  % T2A was declared as font encoding
    \substitutefont{T2A}{\rmdefault}{cmr}
    \substitutefont{T2A}{\sfdefault}{cmss}
    \substitutefont{T2A}{\ttdefault}{cmtt}
  \fi
\else
% X2 was declared as font encoding
  \substitutefont{X2}{\rmdefault}{cmr}
  \substitutefont{X2}{\sfdefault}{cmss}
  \substitutefont{X2}{\ttdefault}{cmtt}
\fi


\usepackage[Sonny]{fncychap}
\ChNameVar{\Large\normalfont\sffamily}
\ChTitleVar{\Large\normalfont\sffamily}
\usepackage{sphinx}

\fvset{fontsize=\small}
\usepackage{geometry}


% Include hyperref last.
\usepackage{hyperref}
% Fix anchor placement for figures with captions.
\usepackage{hypcap}% it must be loaded after hyperref.
% Set up styles of URL: it should be placed after hyperref.
\urlstyle{same}

\addto\captionsfrench{\renewcommand{\contentsname}{Sommaire:}}

\usepackage{sphinxmessages}
\setcounter{tocdepth}{1}



\title{pygame\sphinxhyphen{}animations}
\date{mai 08, 2021}
\release{1.0.0}
\author{K39}
\newcommand{\sphinxlogo}{\vbox{}}
\renewcommand{\releasename}{Version}
\makeindex
\begin{document}

\ifdefined\shorthandoff
  \ifnum\catcode`\=\string=\active\shorthandoff{=}\fi
  \ifnum\catcode`\"=\active\shorthandoff{"}\fi
\fi

\pagestyle{empty}
\sphinxmaketitle
\pagestyle{plain}
\sphinxtableofcontents
\pagestyle{normal}
\phantomsection\label{\detokenize{index::doc}}


\sphinxAtStartPar
Ce paquet est une extension pour pygame qui vous permet d’animer presque n’importe quoi.
\begin{description}
\item[{Installation :}] \leavevmode
\sphinxAtStartPar
\sphinxcode{\sphinxupquote{python3 \sphinxhyphen{}m pip install pygame\sphinxhyphen{}animations}}

\end{description}


\chapter{Comment utiliser le paquet}
\label{\detokenize{usage:comment-utiliser-le-paquet}}\label{\detokenize{usage::doc}}

\section{Intégrer le paquet avec pygame}
\label{\detokenize{usage:integrer-le-paquet-avec-pygame}}

\subsection{Importer le paquet}
\label{\detokenize{usage:importer-le-paquet}}
\sphinxAtStartPar
\sphinxstyleemphasis{Après} avoir importé \sphinxcode{\sphinxupquote{pygame}}, importez le paquet avec

\begin{sphinxVerbatim}[commandchars=\\\{\}]
\PYG{k+kn}{import} \PYG{n+nn}{pygame\PYGZus{}animations}
\end{sphinxVerbatim}

\sphinxAtStartPar
Le nom est un peut long, alors il vaut mieux que vous l’importiez en tant que

\begin{sphinxVerbatim}[commandchars=\\\{\}]
\PYG{k+kn}{import} \PYG{n+nn}{pygame\PYGZus{}animations} \PYG{k}{as} \PYG{n+nn}{anim}
\end{sphinxVerbatim}


\subsection{Appeler \sphinxstyleliteralintitle{\sphinxupquote{update\_animation}}}
\label{\detokenize{usage:appeler-update-animation}}
\sphinxAtStartPar
La seule chose nécéssaire pour que le paquet fonctionne, c’est d’appeler

\begin{sphinxVerbatim}[commandchars=\\\{\}]
\PYG{n}{anim}\PYG{o}{.}\PYG{n}{update\PYGZus{}animations}\PYG{p}{(}\PYG{p}{)}
\end{sphinxVerbatim}

\sphinxAtStartPar
à chaque frame. Le mieux est de l’appeler juste avant de dessiner quoit que ce soit.


\section{Créer des animations}
\label{\detokenize{usage:creer-des-animations}}
\sphinxAtStartPar
Pour créer une animation, créez un objet \sphinxcode{\sphinxupquote{pygame\_animations.Animation}} et appelez sa méthode \sphinxcode{\sphinxupquote{start()}} pour l’éxecuter.


\subsection{Exemple : Déplacer un lutin}
\label{\detokenize{usage:exemple-deplacer-un-lutin}}
\sphinxAtStartPar
Le code suivant déplace le lutin \sphinxcode{\sphinxupquote{monlutin}} en x=200 en 2 secondes :

\begin{sphinxVerbatim}[commandchars=\\\{\}]
\PYG{n}{anim}\PYG{o}{.}\PYG{n}{Animation}\PYG{p}{(}\PYG{n}{monlutin}\PYG{p}{,} \PYG{l+m+mi}{2}\PYG{p}{,} \PYG{n}{rect\PYGZus{}\PYGZus{}x}\PYG{o}{=}\PYG{l+m+mi}{200}\PYG{p}{)}\PYG{o}{.}\PYG{n}{start}\PYG{p}{(}\PYG{p}{)}
\end{sphinxVerbatim}

\sphinxAtStartPar
\sphinxcode{\sphinxupquote{monlutin}} est l’objet à animer. Pour cet exemple, c’est un \sphinxcode{\sphinxupquote{pygame.sprite.Sprite}} mais vous pouvez cibler n’importe quel objet.

\sphinxAtStartPar
\sphinxcode{\sphinxupquote{2}} est la durée de l’animation, en secondes. Vous pouvez aussi passer un flottant, et la durée sera arrondie à la milliseconde.

\sphinxAtStartPar
\sphinxcode{\sphinxupquote{rect\_\_x=200}} est la propriété à animer (ici \sphinxcode{\sphinxupquote{monlutin.rect.x}}).
Le \sphinxcode{\sphinxupquote{\_\_}} remplace les \sphinxcode{\sphinxupquote{.}} pour viser les sous\sphinxhyphen{}propriétés, par exemple \sphinxcode{\sphinxupquote{a.b.c}} devient \sphinxcode{\sphinxupquote{a\_\_b\_\_c}}.
Vous pouvez animer autant de propriétes que vous voulez en même temps.


\section{Synthèse : un « Hello World »}
\label{\detokenize{usage:synthese-un-hello-world}}
\begin{sphinxVerbatim}[commandchars=\\\{\}]
\PYG{k+kn}{import} \PYG{n+nn}{pygame}
\PYG{k+kn}{import} \PYG{n+nn}{pygame\PYGZus{}animations} \PYG{k}{as} \PYG{n+nn}{anim}

\PYG{n}{pygame}\PYG{o}{.}\PYG{n}{init}\PYG{p}{(}\PYG{p}{)}

\PYG{n}{surface} \PYG{o}{=} \PYG{n}{pygame}\PYG{o}{.}\PYG{n}{display}\PYG{o}{.}\PYG{n}{set\PYGZus{}mode}\PYG{p}{(}\PYG{p}{(}\PYG{l+m+mi}{640}\PYG{p}{,} \PYG{l+m+mi}{480}\PYG{p}{)}\PYG{p}{)}

\PYG{n}{font} \PYG{o}{=} \PYG{n}{pygame}\PYG{o}{.}\PYG{n}{font}\PYG{o}{.}\PYG{n}{SysFont}\PYG{p}{(}\PYG{l+s+s1}{\PYGZsq{}}\PYG{l+s+s1}{default}\PYG{l+s+s1}{\PYGZsq{}}\PYG{p}{,} \PYG{l+m+mi}{52}\PYG{p}{)}

\PYG{k}{class} \PYG{n+nc}{MySprite} \PYG{p}{(}\PYG{n}{pygame}\PYG{o}{.}\PYG{n}{sprite}\PYG{o}{.}\PYG{n}{Sprite}\PYG{p}{)}\PYG{p}{:}
    \PYG{k}{def} \PYG{n+nf+fm}{\PYGZus{}\PYGZus{}init\PYGZus{}\PYGZus{}}\PYG{p}{(}\PYG{n+nb+bp}{self}\PYG{p}{)}\PYG{p}{:}
        \PYG{n+nb}{super}\PYG{p}{(}\PYG{p}{)}\PYG{o}{.}\PYG{n+nf+fm}{\PYGZus{}\PYGZus{}init\PYGZus{}\PYGZus{}}\PYG{p}{(}\PYG{p}{)}
        \PYG{n+nb+bp}{self}\PYG{o}{.}\PYG{n}{image} \PYG{o}{=} \PYG{n}{font}\PYG{o}{.}\PYG{n}{render}\PYG{p}{(}\PYG{l+s+s2}{\PYGZdq{}}\PYG{l+s+s2}{Hello, World!}\PYG{l+s+s2}{\PYGZdq{}}\PYG{p}{,} \PYG{l+m+mi}{1}\PYG{p}{,} \PYG{p}{(}\PYG{l+m+mi}{255}\PYG{p}{,} \PYG{l+m+mi}{255}\PYG{p}{,} \PYG{l+m+mi}{255}\PYG{p}{)}\PYG{p}{)}
        \PYG{n+nb+bp}{self}\PYG{o}{.}\PYG{n}{rect} \PYG{o}{=} \PYG{n+nb+bp}{self}\PYG{o}{.}\PYG{n}{image}\PYG{o}{.}\PYG{n}{get\PYGZus{}rect}\PYG{p}{(}\PYG{p}{)}
\PYG{n}{label} \PYG{o}{=} \PYG{n}{MySprite}\PYG{p}{(}\PYG{p}{)}
\PYG{n}{group} \PYG{o}{=} \PYG{n}{pygame}\PYG{o}{.}\PYG{n}{sprite}\PYG{o}{.}\PYG{n}{Group}\PYG{p}{(}\PYG{n}{label}\PYG{p}{)}

\PYG{n}{a} \PYG{o}{=} \PYG{n}{anim}\PYG{o}{.}\PYG{n}{Animation}\PYG{p}{(}\PYG{n}{label}\PYG{p}{,} \PYG{l+m+mi}{2}\PYG{p}{,} \PYG{n}{anim}\PYG{o}{.}\PYG{n}{Effects}\PYG{o}{.}\PYG{n}{cubic\PYGZus{}in\PYGZus{}out}\PYG{p}{,} \PYG{n}{rect\PYGZus{}\PYGZus{}x}\PYG{o}{=}\PYG{l+m+mi}{640}\PYG{o}{\PYGZhy{}}\PYG{n}{label}\PYG{o}{.}\PYG{n}{rect}\PYG{o}{.}\PYG{n}{w}\PYG{p}{,} \PYG{n}{rect\PYGZus{}\PYGZus{}y}\PYG{o}{=}\PYG{l+m+mi}{480}\PYG{o}{\PYGZhy{}}\PYG{n}{label}\PYG{o}{.}\PYG{n}{rect}\PYG{o}{.}\PYG{n}{h}\PYG{p}{)}

\PYG{n}{clock} \PYG{o}{=} \PYG{n}{pygame}\PYG{o}{.}\PYG{n}{time}\PYG{o}{.}\PYG{n}{Clock}\PYG{p}{(}\PYG{p}{)}
\PYG{n}{running} \PYG{o}{=} \PYG{k+kc}{True}

\PYG{k}{while} \PYG{n}{running}\PYG{p}{:}
    \PYG{k}{for} \PYG{n}{ev} \PYG{o+ow}{in} \PYG{n}{pygame}\PYG{o}{.}\PYG{n}{event}\PYG{o}{.}\PYG{n}{get}\PYG{p}{(}\PYG{p}{)}\PYG{p}{:}
        \PYG{k}{if} \PYG{n}{ev}\PYG{o}{.}\PYG{n}{type} \PYG{o}{==} \PYG{n}{pygame}\PYG{o}{.}\PYG{n}{QUIT}\PYG{p}{:}
            \PYG{n}{running} \PYG{o}{=} \PYG{k+kc}{False}

    \PYG{n}{t} \PYG{o}{=} \PYG{n}{pygame}\PYG{o}{.}\PYG{n}{time}\PYG{o}{.}\PYG{n}{get\PYGZus{}ticks}\PYG{p}{(}\PYG{p}{)}
    \PYG{k}{if} \PYG{n}{t}\PYG{o}{\PYGZgt{}}\PYG{l+m+mi}{3000} \PYG{o+ow}{and} \PYG{n}{a}\PYG{o}{.}\PYG{n}{can\PYGZus{}run}\PYG{p}{(}\PYG{p}{)}\PYG{p}{:} \PYG{c+c1}{\PYGZsh{} l\PYGZsq{}animation démarre après 3s}
    \PYG{n}{a}\PYG{o}{.}\PYG{n}{start}\PYG{p}{(}\PYG{p}{)}

    \PYG{n}{anim}\PYG{o}{.}\PYG{n}{update\PYGZus{}animations}\PYG{p}{(}\PYG{p}{)}

    \PYG{n}{surface}\PYG{o}{.}\PYG{n}{fill}\PYG{p}{(}\PYG{p}{(}\PYG{l+m+mi}{0}\PYG{p}{,} \PYG{l+m+mi}{0}\PYG{p}{,} \PYG{l+m+mi}{0}\PYG{p}{)}\PYG{p}{)}
    \PYG{n}{group}\PYG{o}{.}\PYG{n}{draw}\PYG{p}{(}\PYG{n}{surface}\PYG{p}{)}

    \PYG{n}{pygame}\PYG{o}{.}\PYG{n}{display}\PYG{o}{.}\PYG{n}{flip}\PYG{p}{(}\PYG{p}{)}
    \PYG{n}{clock}\PYG{o}{.}\PYG{n}{tick}\PYG{p}{(}\PYG{l+m+mi}{30}\PYG{p}{)}

\PYG{n}{pygame}\PYG{o}{.}\PYG{n}{quit}\PYG{p}{(}\PYG{p}{)}
\end{sphinxVerbatim}


\chapter{Fonctions}
\label{\detokenize{funcs:fonctions}}\label{\detokenize{funcs::doc}}\index{fonction de base@\spxentry{fonction de base}!pygame\_animations.update\_animations()@\spxentry{pygame\_animations.update\_animations()}}\index{pygame\_animations.update\_animations()@\spxentry{pygame\_animations.update\_animations()}!fonction de base@\spxentry{fonction de base}}

\begin{fulllineitems}
\phantomsection\label{\detokenize{funcs:pygame_animations.update_animations}}\pysiglinewithargsret{\sphinxcode{\sphinxupquote{pygame\_animations.}}\sphinxbfcode{\sphinxupquote{update\_animations}}}{}{}
\sphinxAtStartPar
Actualise toutes les animations en cours. Il faut l’appeler avant de dessiner chaque frame.
\begin{quote}\begin{description}
\item[{Paramètres}] \leavevmode
\sphinxAtStartPar
Aucuns

\item[{Renvoie}] \leavevmode
\sphinxAtStartPar
\sphinxcode{\sphinxupquote{None}}

\end{description}\end{quote}

\end{fulllineitems}

\index{fonction de base@\spxentry{fonction de base}!pygame\_animations.stop\_all()@\spxentry{pygame\_animations.stop\_all()}}\index{pygame\_animations.stop\_all()@\spxentry{pygame\_animations.stop\_all()}!fonction de base@\spxentry{fonction de base}}

\begin{fulllineitems}
\phantomsection\label{\detokenize{funcs:pygame_animations.stop_all}}\pysiglinewithargsret{\sphinxcode{\sphinxupquote{pygame\_animations.}}\sphinxbfcode{\sphinxupquote{stop\_all}}}{}{}
\sphinxAtStartPar
Arrête toutes les animations en cours. Voir {\hyperref[\detokenize{animation:stopmethod}]{\sphinxcrossref{\DUrole{std,std-ref}{Animation.stop}}}} pour plus d’informations.
\begin{quote}\begin{description}
\item[{Paramètres}] \leavevmode
\sphinxAtStartPar
Aucuns

\item[{Renvoie}] \leavevmode
\sphinxAtStartPar
\sphinxcode{\sphinxupquote{None}}

\end{description}\end{quote}

\end{fulllineitems}

\index{fonction de base@\spxentry{fonction de base}!pygame\_animations.cancel\_all()@\spxentry{pygame\_animations.cancel\_all()}}\index{pygame\_animations.cancel\_all()@\spxentry{pygame\_animations.cancel\_all()}!fonction de base@\spxentry{fonction de base}}

\begin{fulllineitems}
\phantomsection\label{\detokenize{funcs:pygame_animations.cancel_all}}\pysiglinewithargsret{\sphinxcode{\sphinxupquote{pygame\_animations.}}\sphinxbfcode{\sphinxupquote{cancel\_all}}}{}{}
\sphinxAtStartPar
Annule toutes les animations en cours. Voir {\hyperref[\detokenize{animation:cancelmethod}]{\sphinxcrossref{\DUrole{std,std-ref}{Animation.cancel}}}} pour plus d’informations.
\begin{quote}\begin{description}
\item[{Paramètres}] \leavevmode
\sphinxAtStartPar
Aucuns

\item[{Renvoie}] \leavevmode
\sphinxAtStartPar
\sphinxcode{\sphinxupquote{None}}

\end{description}\end{quote}

\end{fulllineitems}

\index{fonction de base@\spxentry{fonction de base}!pygame\_animations.fastforward\_all()@\spxentry{pygame\_animations.fastforward\_all()}}\index{pygame\_animations.fastforward\_all()@\spxentry{pygame\_animations.fastforward\_all()}!fonction de base@\spxentry{fonction de base}}

\begin{fulllineitems}
\phantomsection\label{\detokenize{funcs:pygame_animations.fastforward_all}}\pysiglinewithargsret{\sphinxcode{\sphinxupquote{pygame\_animations.}}\sphinxbfcode{\sphinxupquote{fastforward\_all}}}{}{}
\sphinxAtStartPar
Termine toutes les animations en cours. Voir {\hyperref[\detokenize{animation:fastforwardmethod}]{\sphinxcrossref{\DUrole{std,std-ref}{Animation.fastforward}}}} pour plus d’informations.
\begin{quote}\begin{description}
\item[{Paramètres}] \leavevmode
\sphinxAtStartPar
Aucuns

\item[{Renvoie}] \leavevmode
\sphinxAtStartPar
\sphinxcode{\sphinxupquote{None}}

\end{description}\end{quote}

\end{fulllineitems}



\chapter{La classe Animation}
\label{\detokenize{animation:la-classe-animation}}\label{\detokenize{animation::doc}}\index{pygame\_animations.Animation (classe de base)@\spxentry{pygame\_animations.Animation}\spxextra{classe de base}}

\begin{fulllineitems}
\phantomsection\label{\detokenize{animation:pygame_animations.Animation}}\pysiglinewithargsret{\sphinxbfcode{\sphinxupquote{class }}\sphinxcode{\sphinxupquote{pygame\_animations.}}\sphinxbfcode{\sphinxupquote{Animation}}}{\emph{target}, \emph{duration}\sphinxoptional{, \emph{effect}}, \emph{**attrs}}{}
\sphinxAtStartPar
Une animation.
\begin{quote}\begin{description}
\item[{Paramètres}] \leavevmode\begin{itemize}
\item {} 
\sphinxAtStartPar
\sphinxstylestrong{target} \sphinxstyleemphasis{(object)}: l’objet à animer.

\item {} 
\sphinxAtStartPar
\sphinxstylestrong{duration} \sphinxstyleemphasis{(int, float)}: durée de l’animation, en secondes.

\item {} 
\sphinxAtStartPar
\sphinxstylestrong{effect} \sphinxstyleemphasis{(callable)}: effet à appliquer à l’animation. Il peut être un {\hyperref[\detokenize{effects:nativeeffects}]{\sphinxcrossref{\DUrole{std,std-ref}{effet natif}}}} ou un {\hyperref[\detokenize{effects:customeffects}]{\sphinxcrossref{\DUrole{std,std-ref}{effet personnalisé}}}}.

\item {} 
\sphinxAtStartPar
\sphinxstylestrong{attrs} : propriétés à animer. pour désigner une sous\sphinxhyphen{}propriété \sphinxcode{\sphinxupquote{a.b.c}}, utilisez \sphinxcode{\sphinxupquote{a\_\_b\_\_c}}

\end{itemize}

\end{description}\end{quote}
\index{target (attribut pygame\_animations.Animation)@\spxentry{target}\spxextra{attribut pygame\_animations.Animation}}

\begin{fulllineitems}
\phantomsection\label{\detokenize{animation:pygame_animations.Animation.target}}\pysigline{\sphinxbfcode{\sphinxupquote{target}}}
\sphinxAtStartPar
(Lecture seule) L’objet ciblé par l’animation.

\end{fulllineitems}

\index{duration (attribut pygame\_animations.Animation)@\spxentry{duration}\spxextra{attribut pygame\_animations.Animation}}

\begin{fulllineitems}
\phantomsection\label{\detokenize{animation:pygame_animations.Animation.duration}}\pysigline{\sphinxbfcode{\sphinxupquote{duration}}}
\sphinxAtStartPar
(Lecture seule) La durée de l’animation, convertie et arrondie en millisecondes.

\end{fulllineitems}

\index{fonction de base@\spxentry{fonction de base}!pygame\_animations.Animation.start()@\spxentry{pygame\_animations.Animation.start()}}\index{pygame\_animations.Animation.start()@\spxentry{pygame\_animations.Animation.start()}!fonction de base@\spxentry{fonction de base}}

\begin{fulllineitems}
\phantomsection\label{\detokenize{animation:pygame_animations.Animation.start}}\pysiglinewithargsret{\sphinxbfcode{\sphinxupquote{start}}}{}{}
\sphinxAtStartPar
Lance l’animation. Elle ne peut être appelée q’une seule fois.
\begin{quote}\begin{description}
\item[{Paramètres}] \leavevmode
\sphinxAtStartPar
Aucuns

\item[{Renvoie}] \leavevmode
\sphinxAtStartPar
\sphinxcode{\sphinxupquote{None}}

\end{description}\end{quote}

\end{fulllineitems}

\phantomsection\label{\detokenize{animation:stopmethod}}\index{fonction de base@\spxentry{fonction de base}!pygame\_animations.Animation.stop()@\spxentry{pygame\_animations.Animation.stop()}}\index{pygame\_animations.Animation.stop()@\spxentry{pygame\_animations.Animation.stop()}!fonction de base@\spxentry{fonction de base}}

\begin{fulllineitems}
\phantomsection\label{\detokenize{animation:pygame_animations.Animation.stop}}\pysiglinewithargsret{\sphinxbfcode{\sphinxupquote{stop}}}{\sphinxoptional{\emph{noerror=False}}}{}
\sphinxAtStartPar
Arrête l’animation et laisse l’objet animé tel quel. Une fois arrêtée, elle ne peut pas être relancée.
\begin{quote}\begin{description}
\item[{Paramètres}] \leavevmode\begin{itemize}
\item {} 
\sphinxAtStartPar
\sphinxstylestrong{noerror} \sphinxstyleemphasis{(bool)}: quand la méthode est appelée sur une animation qui n’est pas en cours, ignore si \sphinxcode{\sphinxupquote{True}} ou lève une \sphinxcode{\sphinxupquote{RuntimeError}} si \sphinxcode{\sphinxupquote{False}}.

\end{itemize}

\item[{Renvoie}] \leavevmode
\sphinxAtStartPar
\sphinxcode{\sphinxupquote{None}}

\end{description}\end{quote}

\end{fulllineitems}

\phantomsection\label{\detokenize{animation:cancelmethod}}\index{fonction de base@\spxentry{fonction de base}!pygame\_animations.Animation.cancel()@\spxentry{pygame\_animations.Animation.cancel()}}\index{pygame\_animations.Animation.cancel()@\spxentry{pygame\_animations.Animation.cancel()}!fonction de base@\spxentry{fonction de base}}

\begin{fulllineitems}
\phantomsection\label{\detokenize{animation:pygame_animations.Animation.cancel}}\pysiglinewithargsret{\sphinxbfcode{\sphinxupquote{cancel}}}{\sphinxoptional{\emph{noerror=False}}}{}
\sphinxAtStartPar
Pareil que {\hyperref[\detokenize{animation:stopmethod}]{\sphinxcrossref{\DUrole{std,std-ref}{stop()}}}}, mais remet l’objet animé dans son état initial.
\begin{quote}\begin{description}
\item[{Paramètre}] \leavevmode\begin{itemize}
\item {} 
\sphinxAtStartPar
\sphinxstylestrong{noerror} \sphinxstyleemphasis{(bool)}: voir {\hyperref[\detokenize{animation:stopmethod}]{\sphinxcrossref{\DUrole{std,std-ref}{stop()}}}}

\end{itemize}

\item[{Renvoie}] \leavevmode
\sphinxAtStartPar
\sphinxcode{\sphinxupquote{None}}

\end{description}\end{quote}

\end{fulllineitems}

\phantomsection\label{\detokenize{animation:fastforwardmethod}}\index{fonction de base@\spxentry{fonction de base}!pygame\_animations.Animation.fastforward()@\spxentry{pygame\_animations.Animation.fastforward()}}\index{pygame\_animations.Animation.fastforward()@\spxentry{pygame\_animations.Animation.fastforward()}!fonction de base@\spxentry{fonction de base}}

\begin{fulllineitems}
\phantomsection\label{\detokenize{animation:pygame_animations.Animation.fastforward}}\pysiglinewithargsret{\sphinxbfcode{\sphinxupquote{fastforward}}}{\sphinxoptional{\emph{noerror=False}}}{}
\sphinxAtStartPar
Pareil que {\hyperref[\detokenize{animation:stopmethod}]{\sphinxcrossref{\DUrole{std,std-ref}{stop()}}}}, mais met l’objet animé dans son état final.
\begin{quote}\begin{description}
\item[{Paramètre}] \leavevmode\begin{itemize}
\item {} 
\sphinxAtStartPar
\sphinxstylestrong{noerror} \sphinxstyleemphasis{(bool)}: voir {\hyperref[\detokenize{animation:stopmethod}]{\sphinxcrossref{\DUrole{std,std-ref}{stop()}}}}

\end{itemize}

\item[{Renvoie}] \leavevmode
\sphinxAtStartPar
\sphinxcode{\sphinxupquote{None}}

\end{description}\end{quote}

\end{fulllineitems}

\index{fonction de base@\spxentry{fonction de base}!pygame\_animations.Animation.is\_running()@\spxentry{pygame\_animations.Animation.is\_running()}}\index{pygame\_animations.Animation.is\_running()@\spxentry{pygame\_animations.Animation.is\_running()}!fonction de base@\spxentry{fonction de base}}

\begin{fulllineitems}
\phantomsection\label{\detokenize{animation:pygame_animations.Animation.is_running}}\pysiglinewithargsret{\sphinxbfcode{\sphinxupquote{is\_running}}}{}{}
\sphinxAtStartPar
Renvoie \sphinxcode{\sphinxupquote{True}} si l’animation est en cours.
\begin{quote}\begin{description}
\item[{Paramètres}] \leavevmode
\sphinxAtStartPar
Aucuns

\item[{Renvoie}] \leavevmode
\sphinxAtStartPar
\sphinxcode{\sphinxupquote{bool}}

\end{description}\end{quote}

\end{fulllineitems}

\index{fonction de base@\spxentry{fonction de base}!pygame\_animations.Animation.can\_run()@\spxentry{pygame\_animations.Animation.can\_run()}}\index{pygame\_animations.Animation.can\_run()@\spxentry{pygame\_animations.Animation.can\_run()}!fonction de base@\spxentry{fonction de base}}

\begin{fulllineitems}
\phantomsection\label{\detokenize{animation:pygame_animations.Animation.can_run}}\pysiglinewithargsret{\sphinxbfcode{\sphinxupquote{can\_run}}}{}{}
\sphinxAtStartPar
Renvoie \sphinxcode{\sphinxupquote{True}} si l’animation n’a pas encore été lancée.
\begin{quote}\begin{description}
\item[{Paramètres}] \leavevmode
\sphinxAtStartPar
Aucuns

\item[{Renvoie}] \leavevmode
\sphinxAtStartPar
\sphinxcode{\sphinxupquote{bool}}

\end{description}\end{quote}

\end{fulllineitems}


\end{fulllineitems}



\chapter{Effets d’animation}
\label{\detokenize{effects:effets-d-animation}}\label{\detokenize{effects::doc}}

\section{Effets natifs}
\label{\detokenize{effects:effets-natifs}}\label{\detokenize{effects:nativeeffects}}
\sphinxAtStartPar
Les effets natif sont regroupés dans l’énumération \sphinxcode{\sphinxupquote{pygame\_animations.Effects}}:


\begin{savenotes}\sphinxattablestart
\centering
\begin{tabulary}{\linewidth}[t]{|T|T|}
\hline

\sphinxAtStartPar
\sphinxcode{\sphinxupquote{Effects.linear}}
&
\sphinxAtStartPar
vitesse constante
\\
\hline
\sphinxAtStartPar
\sphinxcode{\sphinxupquote{Effects.sin\_in}}

\sphinxAtStartPar
\sphinxcode{\sphinxupquote{Effects.square\_in}}

\sphinxAtStartPar
\sphinxcode{\sphinxupquote{Effects.cubic\_in}}

\sphinxAtStartPar
\sphinxcode{\sphinxupquote{Effects.quad\_in}}
&
\sphinxAtStartPar
accélération, de la plus douce à la plus brutale
\\
\hline
\sphinxAtStartPar
\sphinxcode{\sphinxupquote{Effects.sin\_out}}

\sphinxAtStartPar
\sphinxcode{\sphinxupquote{Effects.square\_out}}

\sphinxAtStartPar
\sphinxcode{\sphinxupquote{Effects.cubic\_out}}

\sphinxAtStartPar
\sphinxcode{\sphinxupquote{Effects.quad\_out}}
&
\sphinxAtStartPar
décélération, de la plus douce à la plus brutale
\\
\hline
\sphinxAtStartPar
\sphinxcode{\sphinxupquote{Effects.sin\_in\_out}}

\sphinxAtStartPar
\sphinxcode{\sphinxupquote{Effects.square\_in\_out}}

\sphinxAtStartPar
\sphinxcode{\sphinxupquote{Effects.cubic\_in\_out}}

\sphinxAtStartPar
\sphinxcode{\sphinxupquote{Effects.quad\_in\_out}}
&
\sphinxAtStartPar
accélération au début et décélération à la fin,
de la plus douce à la plus brutale
\\
\hline
\sphinxAtStartPar
\sphinxcode{\sphinxupquote{Effects.sin\_shake}}
&
\sphinxAtStartPar
avance et recule avec une amplitude décroissante
\\
\hline
\sphinxAtStartPar
\sphinxcode{\sphinxupquote{Effects.bounce\_in}}
&
\sphinxAtStartPar
rebondis au début
\\
\hline
\sphinxAtStartPar
\sphinxcode{\sphinxupquote{Effects.bounce\_out}}
&
\sphinxAtStartPar
rebondis à la fin
\\
\hline
\sphinxAtStartPar
\sphinxcode{\sphinxupquote{Effects.bounce\_in\_out}}
&
\sphinxAtStartPar
rebondis au début et à la fin
\\
\hline
\end{tabulary}
\par
\sphinxattableend\end{savenotes}

\begin{sphinxadmonition}{tip}{Astuce:}
\sphinxAtStartPar
\sphinxcode{\sphinxupquote{Effects}} étant une énumération, vous pouvez l’utiliser dans un boucle \sphinxcode{\sphinxupquote{for in}}, et l’effet \sphinxcode{\sphinxupquote{Effects.linear}} peut aussi être désigné par \sphinxcode{\sphinxupquote{Effects{[}"linear"{]}}} ou \sphinxcode{\sphinxupquote{Effects{[}0{]}}} (attention, ils ne sont pas dans l’ordre dans le tableau ci\sphinxhyphen{}dessus).
\end{sphinxadmonition}


\section{Effets personnalisés}
\label{\detokenize{effects:effets-personnalises}}\phantomsection\label{\detokenize{effects:customeffects}}
\sphinxAtStartPar
Un objet qui valide les conditions suivantes peut être utilisée comme effet:
\begin{itemize}
\item {} 
\sphinxAtStartPar
appellable (fonction, expression lambda, …)

\item {} 
\sphinxAtStartPar
prends un seul paramètre, de type \sphinxcode{\sphinxupquote{float}} entre 0 (inclus) et 1 (inclus)

\item {} 
\sphinxAtStartPar
renvoie un \sphinxcode{\sphinxupquote{float}}

\end{itemize}

\sphinxAtStartPar
Le paramètre est la progression \sphinxstyleemphasis{dans le temps} de l’animation (de 0 au début à 1 à la fin).

\sphinxAtStartPar
La valeur revoyée est la progression dans l’animation (0 = état initial et 1 est l’état final). Elle doit être à 0 au début et à 1 à la fin (exception: \sphinxcode{\sphinxupquote{Effects.sin\_shake}} qui revient à 0).



\renewcommand{\indexname}{Index}
\printindex
\end{document}